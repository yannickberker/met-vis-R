% Based on
% - AMA-lato.zip (1cdc979518c99529e679d9a6fefbae8a443236c0)
% - AMA-lato_New.zip (21aa37afca10078d881c2e7c6d0bf4965376eff4)
% from https://analyticalsciencejournals.onlinelibrary.wiley.com/hub/journal/10991492/homepage/latex_class_file.htm

% Begin my commands
\RequirePackage{etex}
\RequirePackage{silence}

\PassOptionsToPackage{colorlinks=true, allcolors=blue}{hyperref}
\RequirePackage{lmodern}

\RequirePackage{pdftexcmds}
\makeatletter
\newcommand{\includeplot}[3][]{%
    \noindent%
    \ifcase\pdf@shellescape\or%
    \immediate\write18{Rscript ../R/plots/#2.R}%
    \or\fi%
    \includegraphics[#1]{#2.pdf}%
    \caption{#3}
    \label{fig:#2}%
}
\makeatother

% https://tex.stackexchange.com/q/94178/
\DeclareRobustCommand*{\citen}[2][]{%
  \begingroup
    \romannumeral-`\x
    \setcitestyle{numbers}%
    \cite[#1]{#2}%
  \endgroup
}
\newcommand{\fcite}[3]{#1 et al.,\cite{#1#2} figure~#3}
\newcommand{\fref}[1]{Figure~\ref{fig:#1}}
\newcommand{\frefs}[2]{Figures~\ref{fig:#1} and~\ref{fig:#2}}
\newcommand{\sref}[1]{section~\ref{sec:#1}}

\RequirePackage{etoolbox}
\newcommand{\highlight}[1]{%
  \expandafter\ifstrequal\expandafter{\jobname}{manuscript-highlight}%
  {{\color{blue}#1}}{#1}}
% End my commands

\documentclass[AMA,LATO1COL]{WileyNJD-v2}

% Begin my commands
\renewcommand{\thefootnote}{\fnsymbol{footnote}}
\graphicspath{{../R/output}}
\RequirePackage{orcidlink}
\newcommand\orcidline[1]{\orcidlink{#1} \url{https://orcid.org/#1}}
\WarningFilter{latex}{Unused global option(s):}
\WarningFilter{latexfont}{Some font shapes were not available}
\WarningFilter{latexfont}{Font shape `T1/lato-TLF/m/sc' undefined}
\WarningFilter{hyperref}{Ignoring empty anchor}
% End my commands

\articletype{Special Issue Research Article}%

\received{16 June 2022}
\revised{14 October 2022}
\accepted{30 October 2022}

\doiheadtext{DOI: 10.1002/nbm.4865}

\raggedbottom

\begin{document}

\title{Visualizing metabolomics data with R}

\author[1,2,3]{Yannick Berker}

\author[4]{Isabella H. Muti}

\author[4]{Leo L. Cheng}

\authormark{BERKER \textsc{et al}}


\address[1]{\orgname{Hopp Children's Cancer Center Heidelberg (KiTZ)}, \orgaddress{\country{Germany}}}

\address[2]{\orgdiv{Clinical Cooperation Unit Pediatric Oncology}, \orgname{German Cancer Research Center (DKFZ) and German Cancer Consortium (DKTK)}, \orgaddress{\country{Germany}}}

\address[3]{\orgname{National Center for Tumor Diseases (NCT) Heidelberg}, \orgaddress{\country{Germany}}}

\address[4]{\orgdiv{Departments of Radiology and Pathology, Harvard Medical School}, \orgname{Massachusetts General Hospital}, \orgaddress{\state{MA}, \country{USA}}}

\corres{Yannick Berker, Clinical Cooperation Unit Pediatric Oncology, German Cancer Research Center (DKFZ), Im Neuenheimer Feld 280, 69120 Heidelberg, Germany. \\\email{yannick.berker@alumni.dkfz.de}}

\presentaddress{Yannick Berker, Siemens Healthineers, Molecular Imaging, 91301 Forchheim, Germany.}

\abstract{In communicating scientific results, convincing data visualization is
of utmost importance. Especially in metabolomics, results based on large numbers
of dimensions and variables necessitate particular attention in order to convey
their message unambiguously to the reader; and in the era of open science,
traceability and reproducibility are becoming increasingly important. This paper
describes the use of the R programming language to visualize published
metabolomics data resulting from ex-vivo NMR spectroscopy and mass spectrometry
experiments with a special focus on reproducibility, including example figures
as well as associated R code for ease of reuse. Examples include various types
of plots (bar plots, swarm plots, and violin plots; volcano plots, heatmaps,
Euler diagrams, Kaplan-Meier survival plots) and annotations (groupings,
intra-group line connections, significance brackets, text annotations).
Advantages of code-generated plots as well as advanced techniques and best
practices are discussed.}

\keywords{data visualization, metabolomics, R}

\jnlcitation{\cname{%
\author{Berker Y},
\author{Muti IH},
\author{Cheng LL}}.
\ctitle{Visualizing metabolomics data with R}, \cjournal{NMR in Biomedicine}, \cyear{2022};e4865. \\doi:\href{https://doi.org/10.1002/nbm.4865}{10.1002/nbm.4865}}

\maketitle

\section{Introduction}

Visualization of experimental data is an important issue in any data science. In
metabolomics, however, we meet specific challenges due to the high
dimensionality of the data: in a usual experiment, hundreds of metabolites each
are quantified for dozens, hundreds, or even thousands of samples and analyzed
using various statistical methods, for groups and subgroups. As a result, the
generation of representations that are accurate, convincing, visually appealing,
and consistent is a complicated endeavor.

There is no shortage of tools to generate almost any kind of plot, ranging from
general-purpose office-suite applications (Microsoft Excel, Apple Numbers, etc.)
to statistics software suites (SAS, JMP, etc.), to dedicated plotting libraries
often programmable via high-level programming languages (Veusz;
\texttt{Matplotlib}, Scikit-plot, and \texttt{plotnine} for Python;
\texttt{ggplot2} for R; MATLAB; etc.). In this tutorial paper, we showcase the
use of the R programming language for data visualization with a focus on
metabolomics applications. We chose R, for it is open source and freely
available, and with \texttt{ggplot2}, \cite{ggplot2} it offers arguably the most
mature implementation of The Grammar of Graphics, \cite{gog} enabling the
separation of data to be plotted from the plot semantics to a high degree. We
will establish necessary terminology through several examples and present best
practices to generate and compose consistent, high-quality, publication-ready
figure files ready for use in posters, presentations, and journal papers without
any postprocessing.

All code used to generate and save the plots in this paper is available for
reuse at \url{https://github.com/yannickberker/met-vis-R}. Functions saving
plots to figure files can be combined with almost any other type of plot:
readers may take further inspiration from websites such as The R Graph Gallery.
\cite{RGraphs}

\subsection{What this tutorial does not cover}

This tutorial focuses on an introduction into technical aspects of data
visualization using examples inspired by metabolomics findings---it does not
intend to cover (metabolomic) data analysis in any way, such as optimal ways of
\highlight{dimensionality reduction,} data stratification and aggregation,
choosing appropriate statistical tests, correcting for multiple testing, or
identifying outliers. We will assume that any data analysis has been implemented
and carried out and, while later amendments of the data are possible, that the
structure of the data to be visualized is determined and the essential data is
given in one data file per plot. For simplicity, we will use data stored in
Microsoft Excel-compatible files for maximum user convenience while maintaining
portability.

While the included code examples may be helpful as a start, implementation
details are not generally a topic of this tutorial. Users should refer to the
link collection titled \textit{Big Book of R}, specifically chapter 12 (Data
Visualization), \cite{BigBookOfR} and ultimately to the reference manuals of R
and its respective packages. These will also be helpful in implementing the
ideas for advanced data representations put forth in \sref{MoreData}.

Using R code for data visualization does not necessarily contradict using an
interactive user interface: users less experienced in writing (R) code may
prefer generating basic plots using a \texttt{Shiny}-based, interactive user
interface offered by the \texttt{esquisse} package, which allows exporting of
not only the generated plot, but also the code to reproduce the code
non-interactively later. While this tutorial does not cover this use, the basic
recommendations apply equally to the code thus generated.

\section{Generating different types of plots in R}

The following examples reproduce plots from the published literature, and we are
indebted to the authors of the respective publications for their most helpful
cooperation. The plots cover various aspects of data visualization, in
particular: types of plots (bar plots, swarm plots, and violin plots; volcano
plots, heatmaps, Euer diagrams, Kaplan-Meier survival plots) and annotations
(groupings, intra-group line connections, significance brackets, text
annotations, colors, etc.).

\subsection{Bar plot}

Bar plots, also known as bar charts/graphs, are used to visualize categorical
data (see \fref{bars}). Categories are typically plotted on the horizontal axis,
and the heights or lengths of the bars correspond to their respective values for
the given measured variable. To assist readers, procedures for using the code at
\url{https://github.com/yannickberker/met-vis-R} to generate the plots in this
article are illustrated there using the example of this bar plot and can be
applied to the others presented in this article.

\begin{figure}
    \centering
    \includeplot[width=3in]{bars}{Bar plot using data from \fcite{Yoneshiro}{2019}{1(b)}}
\end{figure}

\subsection{Swarm plot}

Swarm plots show the distribution of \highlight{a continuous numerical attribute
across another, categorical attribute}. Swarm plots include all data points, so
they are best used for relatively small data sets. See \fref{swarm}.

\begin{figure}
    \centering
    \includeplot{swarm}{Swarm plot using data from \fcite{Hagan}{2019}{S3}}
\end{figure}

\subsection{Violin plots}

\highlight{Similar to swarm plots,} violin plots are used to plot
\highlight{continuous} numerical data \highlight{across different categories}.
They are similar to box plots in that they show the distribution of the data,
but they also include the density of each variable and therefore show peaks in
the data. The wider sections of the violin plot show areas of higher data point
density while the narrower sections represent a lower probability of having a
data point. See \fref{violin_swarm} \highlight{for an overlaid comparison of
violin and swarm plots.}

\begin{figure}
    \centering
    \includeplot[]{violin_swarm}{Violin/swarm plot using data from \fcite{Hagan}{2019}{S2C}}
\end{figure}

\subsubsection{Volcano plot}

\highlight{Scatter plots are commonly used to plot two paired continuous
numerical attributes, and used to visualize correlations. They are also ideally
suited to visualize the result of dimensionality reduction techniques such as
principal component analysis, t-SNE, \cite{vanderMaaten2008} or UMAP.
\cite{McInnes2018}}

A volcano plot is a \highlight{special} scatter plot that depicts significance
(e.g., p value) versus magnitude of change (e.g. fold change) from a statistical
test. Volcano plots are useful in that they show large changes within the data
set that have statistical significance. The negative logarithm of the p value is
typically plotted on the \highlight{vertical} (y-)axis, and the
\highlight{horizontal} (x-)axis depicts the logarithm of the fold change. As a
result, more significant data points with lower p values are represented at the
top of the plot. Data points on the far left and right of the plot are also of
interest, as they represent points with large fold changes. See \fref{volcano}.

\begin{figure}
    \centering
    \includeplot[width=4in]{volcano}{Volcano plot using data from \fcite{Wang}{2021}{1c}}
\end{figure}

\subsection{Heatmap}

Heatmaps are two-dimensional plots that represent the magnitude of a data
association using differences in color and intensity. In a typical cluster
heatmap, data associations are visualized as a matrix of cells, with each row
and column representing a category, and each cell is shaded with a color and
intensity that corresponds to an association level. See \fref{heatmap}.

\begin{figure}
    \centering
    \includeplot[angle=90,width=6in]{heatmap}{Heatmap using data from \fcite{Rom}{2020}{6}}
\end{figure}

\highlight{Heatmaps provide a unique opportunity to visualize the results of
correlation analyses. In a typical setting, a \textit{correlation heatmap} is
calculated by computing correlation coefficients between pairs of attributes A
and B along all samples; correlation strengths are typically color-coded. Such a
correlation heatmap is typically symmetric as correlation coefficients do not
consider the order of their arguments, and each half of such a correlation
heatmap carries the full amount of information.

This concept can be extended in metabolomics, where paired samples from the same
subject (e.g., serum and tissue) are often studied. The aforementioned symmetry
can be broken by computing correlation coefficient across samples, namely,
correlating attribute A from tissue samples with attribute B from serum samples,
and vice versa, providing an additional correlation-heatmap and thus, an
additional layer of visualization.}

\subsection{Euler diagrams}

A Euler diagram is used to represent the relationships between sets of data.
They are similar to Venn diagrams in that they have overlapping regions, which
show the shared characteristics between sets. They are distinct from Venn
diagrams, however, in that they do not have to show an entire set. They are thus
useful for representing complex data sets in a simplified manner. See
\fref{euler}.

Of special note, comparing with \fcite{Shen}{2020}{S4} our current plot reflects
both the relationships and area proportions between subgroups in the plots,
while the original plots are only illustrative of the area proportions between
groups.

\begin{figure}
    \centering
    \includeplot[width=4in]{euler}{Euler diagram using data from \fcite{Shen}{2020}{S4}}
\end{figure}

\subsection{Kaplan-Meier plot}

The Kaplan-Meier plot is used to represent the estimated survival function from
a data set. In practice, it is often used in medical research to show the
proportion of patients who survive a certain amount of time with a disease with
or without progression. The plot is visualized as a curve of declining
horizontal steps that approach the survival function. See \fref{kaplan_meier}.

\begin{figure}
    \centering
    \includeplot[width=4in]{kaplan_meier}{Kaplan-Meier plot using data from \fcite{Agathocleous}{2017}{5(a,b)}}
\end{figure}

\highlight{\section{Combining plots}

Visual representations often benefit from juxtaposition of related information,
to emphasize contrast or similarity. As such, \textit{grids} are a way of
automatically combining subplots, giving the authors great control over
placement and arrangement of multiple plots that will usually survive the
copyediting process. See \fref{grid} for an example combining three of the
previous figures; notably, the grid code reuses the code of the individual
figures, avoiding code duplication and hence potential inconsistencies; and each
individual figure is laid out according to the new geometry without stretching
or squeezing. Similarly, \fref{euler} sets two plots side by side using the
\texttt{cowplot} package and \frefs{swarm}{violin_swarm} demonstrate the concept
of \texttt{ggplot2} \textit{facets} (compare with the following section).}

\begin{figure}
    \centering
    \includeplot[width=\textwidth]{grid}{Grid combining the following figures. A,~\fref{swarm}. B,~\fref{bars}. C,~\fref{volcano}}
\end{figure}

\section{Advanced data representations}
\label{sec:MoreData}

In addition to the example plots showcased above, which usually direct the
reader's attention towards a single aspect of the underlying data, several
advanced techniques shall also be mentioned.

\highlight{As shown above}, grids are one way to produce pixel-perfect
arrangements and compositions of subfigures in one figure panel within R,
skipping the time-consuming manual composition (which is also error-prone,
especially in case of updates to subplots). Using similar techniques,
\textit{inserts} can be created, effectively embedding a smaller plot in the
unused space of another.

\highlight{Automating the concept of grids,} facets \highlight{(in
\texttt{ggplot2} terminology)} are a way \highlight{of repeating} a plot for
several \highlight{to many} subgroups of the data, as indicated in
\frefs{swarm}{violin_swarm} for two subgroups, respectively. While this use of
multifaceted figures seems to be rare in some journals where space constraints
are applied, they are a great way of inspecting longitudinal data by
distributing them along one dimension while keeping all other aesthetics (e.g.,
axes, colors, etc.) consistent.

Finally, \textit{interactive} and \textit{dynamic} plots can be created to allow
the reader to interact with larger, multidimensional data, allowing inspection
of individual data points in detail or selecting facets on the fly. These kinds
of plots are usually hosted online as a supplement to static visualization in a
manuscript.

\highlight{\section{Exporting figures}
\label{sec:Export}

Exporting figures in the format requested by outlet guidelines (which include
the file format, figure dimensions, and figure resolution) can be a struggle in
and of itself. Fortunately, R provides a host of graphic devices in the built-in
\texttt{grDevices} package, natively supporting the export to vector (PDF, PS,
SVG) or bitmap (PNG, JPEG, BMP, TIFF) formats. Alternative graphic devices are
available via the \texttt{Cairo} or \texttt{ragg} packages, for instance.}

\highlight{The code provided with this manuscript provides helper functions
\texttt{(gg)save\textunderscore{}cairo\textunderscore{}pdf\textunderscore{}and\textunderscore{}png}
to export plots at chosen dimensions and resolution in both PDF and PNG formats,
covering the most popular of vector and bitmap formats, respectively.}

\section{Best practices and advantages}

Best practices for using R code to visualize metabolic data rest on the usual
best practices for writing computer programming code, including coding style and
version control.

Code should follow a style guide -- such as in the case of R code, the one put
forth by \texttt{tidyverse}. \cite{tidyverse_style} Users choosing to adhere to
this style are supported by packages such as \texttt{styler} \cite{styler} to
automatically reformat code and \texttt{lintr} \cite{lintr} to perform automated
checks of conformance (especially for topics not covered by \texttt{lintr}, such
as syntactically and logically incorrect code). While the benefit for a single
plot author may seem small, other readers of the code -- including the author's
future self -- will be grateful. Remember that \textit{code is written once but
read many times}.\footnote{While it is challenging to attribute this statement
to one author, the basic idea dates back to at least the 1980s, when ``emphasis
was placed on program readability over ease of writing'' already. \cite{ada83}}
Along the same lines, code should be commented to ease future reading and reuse,
keeping in mind that code comments should not (only) describe \textit{what} the
code is doing, but also \textit{why}.

Just like other unformatted text files, \texttt{.R} files are useful to keep
under version control, for example, using systems such as the de-facto standard
\texttt{git}(\url{https://git-scm.com/}). The decentralized, server-less nature
of \texttt{git} allows migrating repositories online or offline at any point in
time, and \texttt{git} repositories can be hosted (with a basic set of
\highlight{attributes}) for free at services such as GitHub.com, GitLab.com, and
others. In fact, \texttt{.R} files to generate the plots in this tutorial are
available at \url{https://github.com/yannickberker/met-vis-R}. While the
learning curve is steep for command-line use of \texttt{git}, integrated
development environments such as RStudio (\url{https://www.rstudio.com})
commonly integrate support for basic \texttt{git} operations through a graphical
user interface.

Developing R code under version control, if used consistently, means that past
versions of plots can be reproduced without keeping a stash of files named
\verb|Plot3_v2_final_YB_LLC_final_larger.png| and the like. Plots of identical
content can later be re-produced in different variants \highlight{(compare with
\sref{Export})}, e.g., at different resolutions and file sizes, or to generate
landscape bitmaps in PNG format for PowerPoint presentations as well as portrait
vector-graphics in PDF format at journal-specific figure dimensions. (Generally,
the author saves every plot in both file formats.) If plots are created with all
proper annotations within R, supposedly final paper figures can be amended in
minutes to resize the resulting figure, or to insert additional data or remove
outliers or patient samples after late retraction of consent (god forbid!). This
becomes an invaluable time-saver, especially if multi-panel figures are composed
within R, removing any need to repeatedly re-compose figures in Acrobat or
PowerPoint.

R is extensible by a universe of general-purposed as well as domain-specific
packages. For maximum reproducibility, reusability, and security, users should
choose packages and package versions that are published in recognized and
reviewed repositories such as CRAN (\url{https://cran.r-project.org/}) or
BioConductor (\url{https://bioconductor.org/}).

\section{Alternatives}

R and \texttt{ggplot2} may be an excellent basis for a data visualization
technology stack, but it is far from being the only choice -- for example, if
research code in written in Python rather than in R. Interfacing is usually
possible by writing data file to disk in commonly-known file formats such as
comma-separated values, or by calling R code from Python processes using
dedicated packages such as \texttt{rpy2} (\url{https://rpy2.github.io/}). Users
who wish to stay within the Python world may prefer using \texttt{matplotlib}
(\url{https://matplotlib.org/}), which features a MATLAB-inspired application
programming interface, or its extension \texttt{plotnine} which applies the
Grammar-of-Graphics concept to Python (while inheriting several of
\texttt{matplotlib}'s inherent limitations).

\section{Conclusions}

Given the vastness of the universe of R packages, R presents itself as an ideal
choice to generate most, if not all, of today's metabolomics data
visualizations, promising a high degree of reproducibility especially if paired
with R-based data processing.

\subsection*{Acknowledgements}

We sincerely thank the authors for sending us their original data presented in
their original papers that allowed us to consider and to reproduce them here
using R: Takeshi Yoneshiro and Shingo Kajimura, Thomas Hagan and Bali Pulendran,
Kivanç Birsoy, Oren Rom and Y. Eugene Chen, Tiannan Guo, Claudia Langenberg,
Noam Bar and Eran Segal. This work was supported in part by the National
Institute on Aging of the National Institutes of Health under award number
R01AG070257. The content is solely the responsibility of the authors and does
not necessarily represent the official views of the National Institutes of
Health. Open Access funding enabled and organized by Projekt DEAL.

\subsection*{Author contributions}
\textbf{Yannick Berker}: methodology; data curation; software; writing---original draft preparation; writing---review and editing. %
\textbf{Isabella H. Muti}: validation; writing---original draft preparation; writing---review and editing. %
\textbf{Leo L. Cheng}: conceptualization; funding acquisition; supervision; resources; writing---original draft preparation; writing---review and editing.

\subsection*{ORCID}
\textit{Yannick Berker}   \orcidline{0000-0002-6707-0834} \\
\textit{Isabella H. Muti} \orcidline{0000-0002-5216-2306} \\
\textit{Leo L. Cheng}     \orcidline{0000-0001-5975-5406}

\bibliography{references}

\end{document}
